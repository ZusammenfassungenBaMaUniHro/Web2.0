 \documentclass{article} %A4
\usepackage[a4paper,left=1.9cm, right=2.1cm,top = 1.2cm,bottom=2.3cm]{geometry}
\usepackage[utf8]{inputenc}%Umlaute
\usepackage[ngerman]{babel} %Texttrennung
\usepackage{graphicx}	%Grafiken
\usepackage{amssymb}
\usepackage{amsmath}
\usepackage{url}

\usepackage[labelformat=empty]{caption}
\title{Zusammenfassung - Web 2.0}
\author{
	Andreas Ruscheinski,
	Marc Meier,
	Christian D.
}

\begin{document}
\maketitle
\tableofcontents
\section{Definitionsversuche}
	\subsection{Social Media, Enterprise 2.0, Web 2.0}
	\begin{itemize}
	\item\glqq Social Media bezeichnen digitale Medien und Technologien (vgl. Social Software), die es Nutzern ermöglichen, sich untereinander auszutauschen und mediale Inhalte einzeln oder in Gemeinschaft zu erstellen.\grqq (Wikipedia)\\
	\item\glqq Enterprise 2.0 bezeichnet im engeren Sinn den Einsatz von sozialer Software zur Projektkoordination, zum Wissensmanagement und zur Innen- und Außenkommunikation in Unternehmen\grqq (Wikipedia)\\
	\item\glqq Web 2.0 ist ein Schlagwort, das für eine Reihe interaktiver und kollaborativer Elemente des Internets, speziell des World Wide Webs, verwendet wird. Dabei konsumiert der Nutzer nicht nur den Inhalt, er stellt als Prosument selbst Inhalt zur Verfügung.\grqq (Wikipedia)
	\item Neue Technologien werden als Enabler gesehen d.h. Technologie ist Grundlange für die Unterstützung im Business Umfeld
	\item Verschiedene Modelle für die Beschreibung von Social Media, Enterprise 2.0, Web 2.0:
		\begin{itemize}
			\item Social Media Honeycomb
			\item SLATES
			\item FLATNESSES
		\end{itemize}
	\end{itemize}
	\subsubsection{Social Media HoneyComb}
	\begin{center}
	\includegraphics[scale=0.5]{img/SMHC_functionality.jpg}
	\includegraphics[scale=0.5]{img/SMHC_implications.jpg}
	\end{center}
	\begin{itemize}
		\item Charackterisierung möglich durch aus Wahl verschiedener Waaben als Baublöcke
		\item Hauptwaabe = Hauptaufgabe, Nebenwaaben = Zusätzliche Möglichkeiten
	\end{itemize}
	\subsubsection{SLATES}
	\glqq SLATES is an initialism that describes the business impacting capabilities, derived from the effective use of Web 2.0 technologies in and across enterprises.\grqq (Wikipedia), Jedes Element beschreibt eine zentrale Komponente von Enterprise 2.0
	\begin{description}
		\item[SEARCH] Unterstütze die unternehmensweite Suche nach Inhalten;
		Finden, was das Unternehmen weiß und in welchen Köpfen
		\item[LINKS] Ähnliches verbinden nach Assoziationen und Gemeinsamkeiten;
		Abkehr vom  "Teile und herrsche"
		\item[AUTHORSHIP] Niedere Barrieren für Authorship;
		Jeder kann und soll sich als Autor kreativ einbringen
		\item[TAGS] Jeder kann dem Wissen seine Strukturen überstülpen;
		Emergente Strukturen lösen starren Hierarchien und Schemata ab
		\item[EXTENSIONS] Erweiterung der eigenen Sicht und Bewertung;
		Empfehlungssysteme, Bewertungsportale
		\item[SIGNALS] Push und Pull durch Signal-basierte Kanäle ersetzen
	\end{description}
	\subsubsection{FLATNESSES}
	FLATNESS ist eine Erweiterung für SLATES basieren auf der Beobachtung von Enterprise 2.0 in Action. Einige Beobachtungen:
	\begin{itemize}
		\item Enterprise 2.0 is going to happen in your organization whether you
		like it or not
		\item Enterprise 2.0 doesn’t seem to put older IT systems out of business
		\item the benefits of Enterprise 2.0 can be dramatic, but only build steadily
		over time
	\end{itemize}
	\begin{center}
		\includegraphics[scale=0.5]{img/FLATNESS.jpg}
	\end{center}
	\subsection{Vergleich Enterprise 1.0 und Enterprise 2.0}
	\begin{center}
		\includegraphics[scale=0.8]{img/CompareW1W2.jpg}
	\end{center}
	\subsection{Konzepte}
	\begin{itemize}
		\item Entwicklung des Web 2.0 führte zu einer Reihe neuer Ansätze
		\begin{description}
			\item[User Interface:] Lightweight Rich Client, Ajax, Desktop Feel
			\item[Content:] User Generated Content, Prosumer
			\item[Cooperation:] Wisdom of Crowd, Crowd Sourcing
			\item[Economy:] Attention, Intention, Network Economy
			\item[Life Cycle:] Perpetual Beta, Consumer designed product
			\item[Social Media:] facebook usw...
		\end{description}
		\item zentrale Merkmale: Interaktion, Kollaboarion, Sharing, User Generated Content
	\end{itemize}
	\subsubsection{Anwendungen - User Generated Content}
	Der Inhalt wird durch den User erstellt. Mögliche Kategorisierung von Web 2.0 Anwendungen:
		\begin{description}
			\item [Upload Content:] Der Benutzer lädt den Inhalt einer Webseite hoch, Bsp: Youtube, Flickr
			\item [Complete User Dominance:] Der Benutzer ist zentrales Element der Webseite, Bsp: Foren, Blogs, Wikis, Twitter
			\item [Trust Themen:] ???
			\item [Kollaboration:] Die Benutzer arbeiten Zusammen, Bsp: Google Docs, Etherpad $\rightarrow$ Prosumers
		\end{description}
	\subsubsection{Anwendungen - Cooperation Portals}
	Ein Unternehmen stellt ein ????. Mögliche Kategorisierung von Coopartion Portale:
		\begin{description}
			\item[Business Networks:] Plattform für den Austausch von unternehmesrelevanten Informationen, Bsp: XING; LinkedIn
			\item[Interest Networks:] Plattform für den Austausch von Interessen der Benutzer, Bsp: Facebook, MySpace
			\item[Synchronization:] Plattform für die Synchronization von Benutzerinformationen über verschiedene Endgeräte, Bsp: Foxmarks(Lesezeichen), Plaxo(Kontaktdaten)
		\end{description}
	\subsubsection{Technologien - Konzepte}
	\begin{description}
		\item[Advanced Javascript:] Einsatz von komlexen Javascript Programmen für die Realisierung der Webseite
		\item[AJAX,XHR,HTML5:]  Realisierung einer dynamischen Webseite durch asynchrone Kommunikation
		\item[Folkonomies und Tags:] Webseite stellt eine Benutzer definierte Möglichkeit für die Organisation der Daten, Bsp: Hashtag, Subreddit
		\item[Browser Extensions und Plugins:] Erweiterung der Browser durch den Anwendungsentwickler, dadurch Desktop Look-and-Feel
		\item[Probabilistische Algortihmen (Bitcoin):] Einsatz von Probablistischen Algorithmen auf verteilter Basis für die Erbrining einer Leistung, Bsp: Bitcoin(Online Wallet)
	\end{description}	
	\subsubsection{Technologien - Werkzeuge}
	\begin{description}
		\item[JS Libraries:] jQuery, extJS, Angular
		\item[Embedded JS:] Einsatz von Javascript auf Server/Desktop Ebene, Bsp: V8, node.js
		\item[Ajax KITS:] Implementierung für die asynchrone Kommunikation, Bsp: Rico, Dojo, GWT
	\end{description}
	\subsection{Notizen}
		\begin{itemize}
			\item Silos and Boundaries
				\begin{itemize}
					\item Problem bei nicht-Silos: Snowden $\rightarrow$ Daten werden selbstständig
					\item Silos sind alles an Daten, die in einem Umfeld formatiert sind, die man aber irgendwann nicht mehr interpretieren kann $\rightarrow$ sie werden inkompatibel
					\item Wünschenswert: Migration von bspw. Facebook zu MySpace
				\end{itemize}
				\item Prosumers: Nutzer in der Doppelrolle \glqq Inhalt von allen für alle\grqq
				\item Die Daten-Sammel-Phase ist zuende $\rightarrow$ wie sollen sie verarbeitet werden? $\rightarrow$ Google weiß bspw. schon vorher ob eine Grippewelle eintritt
		\end{itemize}
		
		
		
	\section{AJAX und Web 2.0}
	\subsection{Allgemein}
	\begin{itemize}
		\item AJAX ist KEINE neue Technologie sondern Design Pattern
		\item Problem: Verhalten des Benutzers anders als bei Desktop Anwendungen
		\begin{itemize}
			\item Blättern durch einzelne Seiten
			\item Wenig Interaktivität 
			\item Kein undo/redo
			\item Viele Unklarheiten - Reload, Bookmarks und zurück
		\end{itemize}
	\end{itemize}
	\subsection{Vergleich der Reload Cycles}
	\begin{center}
		\includegraphics[scale=0.5]{img/HTTP_Ajax.png}
	\end{center}
	\subsection{AjaX}
	\begin{itemize}
		\item AjaX = Asynchronous Javascript XMLHTTPRequest (XHR)
		\item Asynchron = GUI nicht blockiert während Übertragung
		\item kein neuer Seitenaufbau sondern dynamische Erweiterung mittels Javascript
		\item Arbeitet als Single-Thread, was kritische Probleme löst (Es muss nicht auf parallele Entwicklung geachtet werden)
	\end{itemize}
	\subsection{Konzepte und Technologien}
	\begin{itemize}
		\item Schwächer gekoppelte Kommunikation
		\begin{itemize}
			\item Asynchrone Kommunikation: XMLHttpRequest, Sockets
			\item Varianten der synchrone Kommunikation: Multipart MIME, Server Sockets
		\end{itemize}
		\item Leichtgewichtiges Processing in n-tier Denkweise
		\begin{itemize}
			\item Client-Side Scripting: Javascript
			\item Server-Side Code Generation: PHP, Java – basierte Ajax Kits
		\end{itemize}
		\item OO Dokumentenmodell mit Repräsentationsabtrennung
		\begin{itemize}
			\item Markup: HTML, XML, XHTML
			\item Repräsentations Transformation: CSS, XSLT
			\item DOM – Manipulation: innerHTML, JS DOM
		\end{itemize}
	\end{itemize}
	\subsection{First Player und Impulse}
	\begin{itemize}
		\item Es wird früh eine Entscheidung gefällt, sodass sich spätere und bessere Lösungen nicht mehr durchsetzen können. Aufkaufen der Konkurrenz sichert die eigene Existenz
		\item typischer Second-Mover: Microsoft
		\item Google Suggest: Interactive GUI
		\item Writely $\rightarrow$ Google Docs
		\item Google Maps und Gmail
		\item Technorati (Realtime Trend Analyse mit humane Klassifikation)
		\item Flickr und Youtube (Prosument)
		\item The Broth (Multiuser Game (Mosaik),JS basiertes Drag-and-Drop)  
		\item LinkedIn
	\end{itemize}
	\subsection{Wirtschaftliche Aspekte}
	Einfach mal in den Folien lesen \glqq02-netzwerk-effekte-2015-bw.pdf\grqq
		\begin{itemize}
			\item Lizensierung\\
			(Office like oder Apps)
			\item Lock up in the Cloud\\
			(Amazon und Weihnachten)
			\item Lock up in the App
			\item Lock up in the App Store
			\item Half-Open Source\\
			(Pro open-Source: entwickelt für mich; Contra: gewisser Kontroll-Verlust)
			\item Affero Lizenz
			\item Mandatory Binary Signing
		\end{itemize}
		\subsection{Geschäftsprobleme im Internet}
			Das Problem:
			\begin{itemize}
				\item Viele (gute) Lösungen sind kostenfrei\\
				(warum was zahlen?)
				\item Metering ist schwierig\\
				(Wie messe/berechne ich Nutzung?)
				\item Brand Building ist extrem teuer (bspw. Apple/Google)
				\item Advertising ist ausgelutscht				
			\end{itemize}
			Die Kernfragen:
			\begin{itemize}
				\item Wer zahlt an wen?
					\begin{itemize}
						\item Oft nicht der unmittelbare Nutznießer der Dienstleistung
						\item Bsp: Google: Es zahlt nicht der Such-Kunde (Finanziert sich durch Aufmerksamkeits-Lenkung)
						\item Bsp: Firefox: Es zahlt nicht der Browser-Kunde (Finanziert sich durch Default Homepage / Search Engine Google)
					\end{itemize}
				\item Warum wird gezahlt? 
					\begin{itemize}
						\item Geleisteter Dienst, aber auch ausgenutzte Bindungen
						\item Bsp: Facebook: Kenntnis der Daten des Endanwenders
						\item Bsp: AppStore: Zwangsbündelung über Installation
					\end{itemize}
				\item Wofür wird gezahlt?
					\begin{itemize}
						\item Für Daten oder Dienste
						\item Für ein Enabling (Zugang, Default-Einstellung, Top Platzierung)
					\end{itemize}
				\item Wie schützt sich das Modell vor Nachahmung?
			\end{itemize}
		Mögliche Lösungen
			\begin{itemize}
				\item Premium Services sind kostenpflichtige high-end Versionen einer Dienstleistung
				\item Commercial Services sind Versionen einer Dienstleistung für den comm./prof. User\\
				(Zuverlässigkeit, Features, Rechtsgarantien)
				\item Technologie-Kompetenz als Outsourcing einer high-end IT Abteilung
				\item Solution Hosting betreibt eine komplexe Technologie Server-seitig
				\item Infomediary betreibt einen Handelsplatz zwischen Bedürfnissen
				\item Verbund-Angebote / Lock-In Angebote
			\end{itemize}
			
			
			
			
	\section{XMLHttpRequest}
	\subsection{Allgemein}
	\begin{itemize}
		\item XMLHttpRequest ist ein JS Objekt, REQUEST öffnet HTTP Verbindung, RESPONSE verwaltet die Antwort
		\item Allgemeines Vorgehen:
		\begin{enumerate}
			\item Neues XMLHttpRequest-Objekt erstellen
			\item Request spezifizieren: TYP(GET,POST,PUT,DELETE,\dots),URL (ggf. mit Query Encoding), Boolean Asyc (True für Asynchrone Kommunikation, False für Synchrone Kommunikation)
			\item Setzen des Event Handlers (eine Funktion), welche Ausgeführt wird wenn Antwort vom Server kommt
			\item Senden des Request ggf. mit Body (bei POST leer)
		\end{enumerate}
		\item GET ist relativ unsicher(leicht einsehbar, gilt nicht bei XHR), ggf. Problem mit Firewalls (Empfohlen wenn wenig Daten im Query)
		\item POSTS werden nicht gecached und nervige Browsermeldung(nicht bei XHR) (Empfohlen bei vielen Daten im Query)
		\item Synchron (nicht Ajax): sinnvoll wenn die Interaktion des Users zu verhinden (Bsp.: online Banking)
		\item Asynchron: Programieraufwand (Timeout, Fehlerfall, Mehrfach-Requests, Anzeige an die User über laufenden Ladevorgang)
		\item Umgang mit Fehlerfällen: Defensiv Programmieren: überall entsprechende Exception Handler vorsehen, Viel Logging, Davon ausgehen,d ass alles irgendwann mal irgendwie schief läuft
	\end{itemize}
			
	
	\section{Fehlendes Kapitel}	
	
		
	\section{Bitcoin}
	\subsection{Allgemein}
	\begin{itemize}
		\item \glqq Geld ist ein \textbf{Recht} zur Ausführung einer bestimmten Transaktion das von seinem \textbf{Träger} \textbf{genau einmal ausgeübt} werden kann und \textbf{nur durch} die Ausübung auf \textbf{andere übergeht}.\grqq
		\item Geld als Wertmaßstab, Tauschmittel und Wertaufbewahrung
		\item Bedrohungen: Wert-Verlust, Inflation, System-Crash,...
		\item \glqq Träger:\grqq
		\begin{itemize}
			\item an Name gebunden
			\item an Kenntnis gebunden
			\item an Pseudonym gebunden
			\item an den Besitz eines Objekts gebunden
			\item an Körper gebunden
		\end{itemize}
		\item \glqq Genau einmal ... von seinem Träger \grqq
		\begin{itemize}
			\item Geld kann nicht mehrfach Ausgegeben
			\item Weitergabe möglich
			\item Backup möglich
		\end{itemize}
		\item \glqq Nur durch\grqq
		\begin{itemize}
			\item Kein Erzeugung des Geldes ohne Deckung
			\item Gegenleistung in Zeit, Energie, Gold (Wert-Deckung)
			\item Gesellschaftlich Befugnis (Regelungs-Deckung)
			\item Gesellschaftliche Akzeptanz (De Facto Deckung)
		\end{itemize}
		\item \glqq auf andere übergeht\grqq
		\begin{itemize}
			\item einer verliert,einer bekommt das recht
		\end{itemize}
		\item Wie entsteht der \glqq Wert\grqq des Geldes
		\begin{itemize}
			\item Geld lässt sich \textbf{unmittelbar konvertieren} in Nahrungsmittel, etc.
			\item Es wird akzeptiert und ist vom Gesetzt als Zahlungsmittel vorgeschrieben 
		\end{itemize}
	\end{itemize}
	\subsection{Was ist Bitcoin?}
	\begin{itemize}
		\item Träger des Geldes
		\begin{itemize}
			\item Bindung an Pseudonym (Bitcoin Adresse)
			\item Nachweis der Befugnis durch Key Paar
			\item Bitcoin Adresse ist hash des public key
			\item Private Key verlieren ist Geld verlieren
			\item Identität durch Key Paar Generierung (256 Bit)
		\end{itemize}
		\item Übertragung
		\begin{itemize}
			\item Peer-to-Peer
			\item jeder Teilnehmer betreibt Bitcoin Knoten
			\item jeder Knoten speichert Konto-Stände aller Adressen
			\item Transaktion an alle Knoten gesandt
			\item Berechtigugn wird durch Signatur überprüft
			\item neue Konto-Stände werden an alle Knoten gesandt
		\end{itemize}
		\item Problem: Inkonsistente Konto-Stände (einige Knoten sind noch nicht über neue Kontostände informiert)
		\item Lösung: Bitcoin Blockketten-Algorithmus
		\item Wichtigste Aspekte von Bitcoin (hypeunabhänig)
			\begin{itemize}
				\item Hochreplizierte Datenbank
				\item Wird nach gewisser Zeit global konsistent
				\item probabilistischer Konsistenz-Begriff
			\end{itemize}
	\end{itemize}
	\subsection{Blockketten-Algorithmus}
	\subsection{Diskussion}
		\begin{itemize}
			\item Warum sollten Leute Blöcke versiegeln wollen?
			\begin{itemize}
				\item Wer einen Block versiegelt, erhält 50 BTC
				\item Alle 4 Jahre wird der Wert halbiert $\rightarrow$ Asymptotisch: Maximal 21 Millionen BTC hergestellt\\				
				$\rightarrow$ Eingebauter Inflations-Schutz
				\item Algorithmus sieht vor: Überweisung wird nur durchgeführt, wenn eine kleine Fee gezahlt wird
			\end{itemize}
			\item Konsequenzen
			\begin{itemize}
				\item Wird das Versiegeln ökonomisch unattraktiv
				\item[$\rightarrow$] dann versiegeln weniger Leute
				\item[$\rightarrow$] dann sinkt die Difficulty
				\item[$\rightarrow$] das Versiegeln kostet weniger CPU
				\item[$\rightarrow$] und wird dadurch wieder billiger
				\item[$\rightarrow$]\textbf{Stabiles, sich selbst regelndes System}
			\end{itemize}
			\item Ist Bitcoin anonym?
			\begin{itemize}
				\item ???
			\end{itemize}
			\item Problem: CAP-Theorem
			\begin{enumerate}
				\item \textbf{C}onsistency: Alle Knoten sehen selbe Daten
				\item \textbf{A}vailability: System bleibt verfügbar bei einzelnen Knoten-Ausfällen
				\item \textbf{P}artition Tolerance: Netz-Partition macht kein Anhalten des Algorithmus erforderlich				
			\end{enumerate}
			\begin{itemize}
				\item In einem verteilten System bekommt man von CAP maximal 2 Eigenschaften hin $\rightarrow$ Bewiesen 2002
				\item Bitcoin garantiert alle 3 \\
				(C aber nur im probabilistischen Sinne)
				\item[$\rightarrow$] Ein inkonsistentes System mit elektronischem Geld
			\end{itemize}
			\item Lösung
			\begin{itemize}
				\item Gewisse Fehlerwahrscheinlichkeit akzeptieren, die im Zeitverlauf exponentiell sinkt
				\item Schadenserwartungswert sinkt, je länger man wartet
				\item[$\rightarrow$] nur praktische Erfüllung von C
			\end{itemize}
			\item Ist Bitcoin sicher? Ja wenn:
			\begin{itemize}
				\item ECC, SHA-256 und RIPEMD sicher ist
				\item Der private Key sicher gespeichert wird
			\end{itemize}
		\end{itemize}
	\subsection{Zusammenfassung nach Khan Academy}
	Dieser Abschnitt stammt von Marc.
	Die Infos wurden den Lehrvideos der Kahn Academy \url{https://www.khanacademy.org/economics-finance-domain/core-finance/money-and-banking/bitcoin} entnommen.
	Alle Unterabschnitte entsprechen im wesentlichen Zusammenfassungen der Videos
	\subsubsection{What is it \& Overview}
	Diese Videos bieten eine Einführung zu Bitcoin.
	Es werden ein paar Grundlagen und Vorteile genannt, diese folgen hier eventuell später.
	\subsubsection{Cryptografic hash functions \& Digital signatures}
	Die Inhalte dieser Videos sollten Grundlagen für Informatiker darstellen.
	Auf das Wichtigste heruntergebrochen sind diese:
	\begin{itemize}
		\item Kryptografische Hash-Funktionen sind mathematische Funktionen, die einen Wert (Message;variable Länge) auf einen anderen Wert (Hash, Digest, Tag;feste Länge) abbilden.
		\item Die wichtigsten Eigenschaften sind effiziente Berechenbarkeit, Kollisionsresistenz, Rückschluss auf Eingabe nicht möglich und Streuung der Werte (1 Bit in Eingabe geändert heißt nicht, dass die Ausgaben ähnlich sind)
		\item Bekannte Vertreter sind MD5 und SHA-256
		\item Zum Digitalen Signieren hat ein Teilnehmer ein public-private-Schlüssel-Paar
		\item Eine zu signierende Nachricht wird gehashed und mit dem privaten Schlüssel verschlüsselt, heraus kommt die digitale Signatur, welche Autor und Nachricht bestätigt.
		\item Zum Überprüfen wird die Signatur mit dem öffentlichen Schlüssel entschlüsselt, der entstehende Wert kann mit dem Hash der empfangenen Nachricht verglichen werden.
		\item (Kleines, da schwieriges/unwahrscheinliches) Problem: Aufgrund von Kollisionen (mehrere Nachrichten können selben Hashwert haben) keine 100-prozentige Sicherheit!
	\end{itemize}
	\subsubsection{Transaction records}
	In diesem Video habe ich noch einige Verständnisprobleme, daher wäre eine Sichtung nochmal gut.
	Insbesondere die Unterscheidung Teilnehmer und Transaktion ist mir irgendwie noch nicht zu 100 Prozent klar.
	\begin{itemize}
		\item Bitcoin-Datenbank speichert nicht die Werte einzelner Konten, sondern viel eher die Transaktionen der Bitcoins
		\item Teilnehmer durch Bitcoin-Adresse oder Pseudonym repräsentiert. 
		Diese(s) entspricht einem öffentlichen Schlüssel (eines privat-public-Key-Paares).
		Das bedeutet, dass so eine Adresse ohne zentrale Instanz selbst generiert werden kann, jedoch auch, dass eine (unwahrscheinliche, da $2^256$ Adressen) Kollisionsmöglichkeit besteht.
		Achtung: Im Skript handelt es sich bei der Bitcoin Adresse um den Hash des Schlüssels!
		\item Das Guthaben eines Teilnehmers setzt sich zusammen aus den vergangenen, noch nicht als ungültig (besseres Wort?) erklärten Transaktionen, für die der Teilnehmer eine Berechtigung (in Form eines private Key) hat.
		Beispiel: Alice hat Bob 10 BTC überwiesen, Carol hat Bob 15 BTC überwiesen.
		Beide Überweisungen sind irgendwo in der Historie zu finden.
		Bob hat 25 BTC.
		\item Bei einer Überweisung von 5 BTC von Bob an Dora schreibt Bob eine Transaktion.
		Diese enthält die Quellen, also beispielsweise die Transaktion von Alice an Bob (die besagt, dass Bob 10 BTC) besitzt; den Empfänger Dora (bzw deren public Key/Bitcoin-Adresse) und den Wert in Höhe von 5 BTC, sowie das Rückgeld an Bob (weil wir von einer 10 BTC Transaktion ausgehen, will Bob noch etwas wiederhaben).
		Die Transaktion wird von Bob signiert und an die Knoten weitergegeben, welche diese bestätigen sollen.
	\end{itemize}
	\subsubsection{Proof-of-work}
	Proof-of-work ist ein Verfahren, welches genutzt wird, um die Blockchain aufzubauen.
	Diese erklärung ist erstmal unabhängig von Bitcoin
	\begin{itemize}
		\item Zweck von Proof-of-work: Um einen Dienst zu nutzen (z.B. einen Block in die Blockchain zu hängen) muss zuerst großer rechnerischer Aufwand betrieben werden.
		Die Überprüfung muss sehr einfach funktionieren.
		\item Idee: Zu einer Challenge c soll ein Proof p gesucht werden, sodass der Hash h(c,p) mit mindestens einer bestimmten Anzahl von 0 beginnt.
		Das ist im Wesentlichen ein Brute-Force-Prozess
		\item Zum Überprüfen, ob p ein Proof zu c ist, muss lediglich h(c,p) ausgerechnet werden.
	\end{itemize}
	\subsubsection{Transaction block chains}
	\begin{itemize}
		\item Im vorletzten Abschnitt wurde eine Transaktion an die Knoten weitergegeben
		\item Die Knoten sammeln unabhängig voneinander Transaktionen und fassen diese zu Blöcken zusammen; das heißt die Transaktionen werden immer paarweise gehashed, bis lediglich ein einzelner Hash für den gesamten Block übrig ist.
		\item Ziel ist es, diesen Block an die Blockchain anzuhängen, also mit dem letzten aktuellen Block zu verknüpfen, welcher wiederum mit seinem Vorgänger verknüpft ist (bis zu einem ursprünglichen Genesis-Block)
		\item Die Verknüpfung von jetzigem Block mit dem aktuellsten Block in der Blockchain ist die Challenge. 
		Zu dieser soll ein Proof (letzter Abschnitt) werden.
		Wie viele Nullen dieser mindestens haben soll, wird vom Netzwerk festgelegt (nächster Abschnitt).
		\item Im Schnitt wird alle 10 Minuten ein solcher proof gefunden.
		Alle Knoten arbeiten an eigenen, unterschiedlichen Blöcken (da diese verschiedene Transaktionen enthalten können).
		\item Finden zwei Knoten (in etwa) gleichzeitig einen Proof zu ihrem Block, werden beide Blöcke an die Blockchain angehängt, es entsteht eine Verzweigung.
		Nachfolgende Blöcke sollen immer an die längste Kette angehängt werden.
		Achtung: Gemeint ist nicht die Anzahl der Blöcke, sondern die schwierigste Kette, also jene, welche die meisten "Anfangsnullen" enthält.
	\end{itemize}
	\subsubsection{The money supply}
	\subsubsection{The security of transaction block chains}
	\section{Ajax und Web 2.0 Sicherheit}
	\subsection{Allgemein}
	\begin{itemize}
		\item Motive der Angreifer
		\begin{itemize}
			\item Ausspähen von Daten
			\item Modifikation von Transaktionsdaten
			\item Installation von Mal- und Spyware
			\item Aktivieren von Bot-Netzen
			\item Schädigung von Personen bzw. Firmen
			\item Erzielen von Netzerk-Effekten
		\end{itemize}
		\item Was kann der Angreifer?
		\begin{itemize}
			\item Surfer eine Webseite des Angreifers laden
			\begin{enumerate}
				\item Javascript des Angreifers kommt zur Ausführung $\rightarrow$ Zugriff auf Client Ressourcen (Cookies, Formulardaten)
				\item Surfer glaubt sich auf einer anderen Seite $\rightarrow$ PIN angegeben, da ja auf Bankseite bin
				\item Cross Site Request auf eingebettete Objekte kommt zur Ausführung $\rightarrow$ Autorisierungs-Cookie wird an nicht-intendierten Link gesendet
				\item Lancieren gezielter Angriffe $\rightarrow$ Image / Video / PDF, das einen Zero Day im PlugIn / Player ausnutzt
			\end{enumerate}
			\item Den Content Provider zu einem Einbinden eigener Scripte verleiten (Problem: Scripte haben maximale Rechte auf der Seite)
			\item Den Content Provider zu einem Setzen von Links verleiten 
			\item Eingabe spezielle formatierten Inputs 
		\end{itemize}
		\item Konzentation auf Cross Site Angriffe
		\item Weitere Probleme: Soziale Probleme des Mitmach Models (Multible Identitäten)
	\end{itemize}
	\subsection{Sicherheitsmodelle für mobilen Code}
	\begin{itemize}
		\item Problem: Mobiler Code = Ausführebare Datei, die vom Server geladen wird
		\item Lösungen: Sandbox, Signierter Code, Policy Files
	\end{itemize}
	\subsubsection{Sandbox}
	\begin{itemize}
		\item Mobiler Code läuft in Sandbox und verhindert dadurch Zugriff auf lokale Ressourcen 
		\item Problem: ggf. zu weite Eingrenzung der Rechte (keine clientseitige Persistenz), kein kontrolliertes Durchbrechnen möglich
	\end{itemize}
	\subsubsection{Signierter Code}
	\begin{itemize}
		\item Code wird durch Instanz signierte, Signierter Code erhält weitergehende Rechte
		\item Problem: Vertrauen an Drittinstitutionen, Unklar was weitergehende Rechte bedeuten, Nicht sinnvoll praktikabel (hoher Aufwand)
	\end{itemize}
	\subsubsection{Policy Files}
	\begin{itemize}
		\item User kann steuern, welche Rechte an das Script freigegeben werden (sehr Flexibel)
		\item Problem: ggf. zu kompliziert für User, nicht zero install
	\end{itemize}
	\subsection{Sicherheit von Javascript - Same Origin Sandbox}
	\subsubsection{Allgemein}
	\begin{itemize}
		\item Eingebundenes Javascript auf einer Seite von Seite X hat Lese/Schreib-Rechte auf den Dokumenten welche die selbe Herkunft (Seite X) haben
		\item selbe Herkunft ist verletzt gdw:
		\begin{itemize}
			\item Verschiedene Domäne in der URL
			\item Protokoll verschieden
			\item Port verschieden
			\item Ein Script von Rechnername, ein anderes von IP Adresse geladen
			\item Unterschiedliche Subdomänen
			\item Redirections
		\end{itemize}
		\item Durchbrechen der Sandbox Benutzerkontrolliert oder Systematisch nach definierten Standard(HTML5 File Reader/Writer)
		\item Schutzziele:
		\begin{itemize}
			\item Lokale Ressourcen: kein Zugriff auf Festplatte, Devices, Sensoren, Durchbruch nur systematisch
			\item Fremde Ressourcen: kein Zugriff auf den DOM/Formulardaten anderer Dokumente
			\item Remote Ressourcen: kein Zugriff fremde Domänen
		\end{itemize}
		\item Problem: ISP-Anomalität (www.myisp.com/dave/script1.js und www.myisp.com/john/script1.js können aufeinander zugreifen), document.domain Property (www.bsp.com und wiki.bsp.com können nicht interagieren $\rightarrow$ document.domain = \glqq bsp.com\grqq)
		\item XHR, localStorage und Web Messaging unter liegen Same Origin Restriktion
		\item Bilder, CSS und Skripte unterliegen \textbf{nicht} Same Origin Restriktion $righarrow$ dadurch Attacken möglich: Web Spoofing, GUI Redressing, Seiteübernehmen
		\item Websockets haben eigene Sicherheitsschicht
		\item Cooklies unterliegen browserseitigen Einstellungen
		\item Flash Cookies unterliegen browser- und Flash-Einstellungen
	\end{itemize}
	\subsubsection{Durchbrechen der Same Origin Sandbox}
	\begin{itemize}
		\item Durchbrechen durch Cross Site Zugriffe (???)
		\begin{itemize} 
			\item Möglich durch: Javascript, CSS, Requests
			\item Wenig Problem da Vermeidung durch geeigneten Webseiten Entwurf
		\end{itemize}
		\item Durchbrechen durch Entscheid des Browser-Herstellers
		\begin{itemize}
			\item Explorer: Durch OS Einstellungen, ActiveX/Flash Komponenten
			\item Firefox: Durch signed code, Plugins, Flash Komponenten
			\item Sollte man nicht nutzen, da plattformabhängig, öfters unsicher implementiert und
			oft geändert, unterschiedliche Mitwirkung des end-users erforderlich
		\end{itemize}
		\item Durchbrechen durch Protokoll-Erweiterung
		\begin{itemize}
			\item CORS = Cross Origin resource Sharing (W3C Standard)
			\item Web Messaging(HTML5)
			\item JsonRequest (???)
			\item JSONP (????)
		\end{itemize}
	\end{itemize}
	\subsubsection{CORS - Cross Origin Resource Sharing}
	\begin{itemize}
		\item ermöglich kontrolliertes Durchbrechen der Same Origin Sandbox durch Beteiligung des Servers
		\item erlaubt XHR auf andere Domäne 
		\item Nutzt ggf. Preflighting (HTTP OPTIONS Request) d.h. holt sicher Erlaubnis des Servers zur Abweichung vom üblichen Verhalten
		\item Preflighting wenn Methode nicht GET oder POST, Request body hat MIME Type != text/plain oder spezielle Header hinzugefügt
		\item Ablauf:
		\begin{enumerate}
			\item Server alice.com hat Daten
			\item Server bob.com will Daten von alice.com einbinden
			\item alice.com versieht Daten mit zusätzlichen Header
			\item Browser gestattet der Seite von bob.com ein XHR auf alice.com
		\end{enumerate}
		\item Alternative: Trust, bob vertraut alice, alice erlaubt bob die Nutzung (Headererweiterung)
		\item Client: alles wie immer :D
		\item Server: diverse neue Header, welche verarbeitet werden müssen
	\end{itemize}
	\begin{center}
		\includegraphics[scale=0.3]{img/CORS_FLOW.png}
	\end{center}
	\subsubsection{Web Messaging}
	\begin{itemize}
		\item Ermöglicht den Datenaustausch zwischen Dokumenten in anderen Tabs/Fenstern
		\item Senden: targetWindow.postMessage(message,targetURL)
		\item targetWindow = Referenz auf Zielfenster (Erhalt durch Öffnen des Fenster $\rightarrow$ return-Wert von window.open(); enthalten als contentWindow Property eines iframes)
		\item message = String oder Object das gesendet wird
		\item targetUrl = URL des Zielfensters
		\item Empfänger implementiert onmessage-Handler $\rightarrow$ erhält ein Event mit den Attributen data (Inhalt der Nachricht), origin (URL des Senders), source (Referenz auf das Sender-Fenster)
		\item Sicherheit
		\begin{itemize}
			\item Sender: Kann vertrauliche Daten an böses Ziel senden 
			\item Empfänger: Erhalte nicht vertrauenswürdige Daten aus böser Quelle
		\end{itemize}
	\end{itemize}
	\subsubsection{iframe Security}
	\begin{itemize}
		\item Inhalte können von Cross-Site über iframes eingebunden werden
		\item Problem: Fremde Site verändert eigene Site, Nutzer sieht nicht wer für den Inhalt verantwortlich, Site macht sich fremde Inhalte zu Nutze
		\item Lösung: sichtbar einbetten, iframe busting (Javascript der eingebetten Seite greift auf parent zu und schreibt sich auf höchster Ebene), Server setzt X-Frame-Options Header auf Deny (komplettes Verbot), Allow-From url (Einbinden wenn Basisseite bestimmte URL ist) oder Sameorigin (Einbinden nur wenn Basisseite same origin ist)
	\end{itemize}
	\subsubsection{Cross Site CSS}
	\begin{itemize}
		\item Einbinden von CSS von andere Seite
		\item Problem: Verdecken von Elementen, Unsichtbarkeit von Elementen, GUI redressing Attacken
	\end{itemize}
	\subsubsection{Cross Site Scripting}
	\begin{itemize}
		\item Definition: Webseite stammt aus einer Domain und lädt Skripte von einer anderen Domain
		\item WICHTIG: Cross Domain XHR verhindert durch Sandbox, Cross Domain Scripting wird nicht verhindert durch Sandbox
		\item Warum? Eigentlich wenig Sinn ausser bei Zugriff auf Mashup (Google Maps, Flickr Gallerie)
		\item Probleme: Fremdes Script zerstört Aussehen meiner Web Site völlig, Fremdes Script liest Werte aus Formularen aus

		\item Bookmarklet = ist ein kleines in JavaScript geschriebenes Makro, das als Bookmark abgespeichert wird und dadurch die Funktionen eines Webbrowsers erweitert; Problem: laufen mit den Rechten der aktuell angezeigten Seite und können so Cookies-Stehlen und remote-Code hinzufügen
	\end{itemize}
	\subsubsection{Cross Site Request Forgery}
	\begin{itemize}
		\item Angreifer plaziert IMG-Link auf einer Seite mit Link welcher aufgerufen wird wenn Seite geladen wird
		\item Bsp: $\le$img src=\glqq http://bank.example/withdraw?account=bob\&amount=1000000\&for=mallory\grqq$\ge$
		\item Verteidigung: 
		\begin{itemize}
			\item nutzen von Logoff (Anfragen werden verworfen da Benutzer nicht angemeldet)
			\item Service nur als POST anbieten
			\item Zur Autorisierung auch ein Hidden Field nutzen
			\item Cookie double submission: Cookie und dazupassenden Teil in einem hidden field erwarten
		\end{itemize}
		\item Probleme:
		\begin{itemize}
			\item Cookies zur Vermittelung von Rechten (bei Aufruf wird Cookie verstandt); Erwünscht: Cross Domain User Tracking (Marketing), Unerwünscht: weil Cookies auch zum Speichern von User Credentials benutzt werden
			\item 2 implizite Annahmen im Link-Konzept: User sieht den Link, auf den er klickt aber er sieht nur Inhalt des Link-Tags nicht die URL $\rightarrow$ Betrug durch falsche GUI Konstruktion; User kennt den Effekt des Links, auf den er klickt aber URL namens abbuchen.php muss nicht abbuchen $\rightarrow$ Betrug durch vorgespiegelte URL Semantik
		\end{itemize}
	\end{itemize}
	\subsection{Konzeptuelle Probleme bei Berechtigungen}
	\subsubsection{Confused Deputy - Problem}
	\begin{itemize}
		\item Eine Komponente hat ein Recht aber das Recht wird genutzt, aber ausserhalb der erwarteten Verarbeitungssequenz oder ausserhalb des erwarteten Verarbeitungskontexts
		\item Eine Komponente referenziert eine andere Komponentee welche genutzt wird aber anders als mit der intendierten Semantik der Komponente
	\end{itemize}
	\subsubsection{Wie funktionieren Berechtigungen?}
	\begin{itemize}
		\item Owner Based Rights
		\begin{itemize}
			\item Objekte gehören einem User
			\item UserId hat bestimmte Rechte auf dem Objekt
			\item Programm unter einen UserId hat bestimmte Zugriffsrechte
			\item Problem: Kann Zugriffsrechte nicht mit anderen User teilen
			\item Lösung: Gruppenkonzept
		\end{itemize}
		\item Group Based Rights
		\begin{itemize}
			\item Objekte haben auch eine Gruppe
			\item User gehören in eine Gruppe
			\item User in einer Gruppe haben bestimmte Rechte
			\item Problem: Gruppenstruktur kann sehr komplex werden
			\item Lösung: Access Control Lists
		\end{itemize}
		\item Access Control Lists (ACL)
		\begin{itemize}
			\item Objekte haben eine ACL
			\item ACL weiss was welcher User darf
			\item Problem: Recht hängt nur vom User ab, nicht vom Programm oder Kontext des Programmes	
		\end{itemize}
		\item Capabilities
		\begin{itemize}
			\item Zugreifende Entitäten haben spezifische Fähigkeiten, auf Objekte zuzugreifen
			\item ???
		\end{itemize}
		\item Kontextuelle Restriktionen
		\begin{itemize}
			\item Bindung an den Kontext / die Seite / den Referer
			\item ???
		\end{itemize}
	\end{itemize}
	\subsubsection{Social Engineering}
	Social Engineering bedeutet den Mißbrauch von Gruppen-spezifischen Verhaltensweisen um den Benutzer zu einer Handlung zu bewegen die er unter Kenntnis der damit verbundenen Folgen so nicht setzen würde
	\subsection{Fehlen von Trusted IO}
	\subsubsection{GUI Redressing Attacke}
	\begin{itemize}
		\item Attacke welche durch eine versteckte Veränderung der Oberfläche erfolgt 
		\item Möglichkeiten: Cursordatei manipulier
		\item Clickjacking: Stehle Mausklick des Users (Angreifer spiegelt ihm vor, dass er auf eine andere Stelle klickt)
		\item Strokejacking: Stehle den Tastendruck des Users (Browser sieht andere Taste als jene die der User gedrückt hat)
	\end{itemize}
	\subsubsection{Web Spoofing Attacke}
	\begin{itemize}
		\item Benutzer ist auf anderem Server als er glaubt
		\item Ziel: User vorzuschleiern auf einer anderen Seite zu sein als er eigentlich ist (User denkt er ist auf bank.de, hat aber mafia.de aufgerufen)
	\end{itemize}
	\subsubsection{Injection Attacken}
	\begin{itemize}
		\item Techniken zur Generierung dynamischer Seiten erhalten Daten vom Client
		\item Techniken werden eingesetzt bei: SQL Queries, HTML Seiten, PHP Seiten, Javascript Programme
		\item Beispiel: Einschleusen von HTML Code auf Webseite durch Formulardaten
		\item Beispiel: Manipulation des SQL-Query durch Manipulation im HTTP-Request
		\item Lösung: Validierung der Eingabe, Escapen von Quotes
	\end{itemize}
	\subsubsection{Covert Channel}
	\begin{itemize}
		\item ???
	\end{itemize}
	\section{Einführung in Javascript}
	\subsection{Allgemein}
	\begin{itemize}
		\item Ziel: Dynamische Seiten ohne Server-Interaktion, Adaptive Webseiten mit rascher Reaktion auf Client-Interaktion, Rechenaufwand auf Client verteilen
		\item Lösungen: Javascript, VB Script, Java Applets, Active X
		\item Später neue Anforderungen: Zugriff auf Sensoren (GPS, Webcam, Mic), Performane Boost durch clientseitige Parallelität und stärkere Kontrolle der User (App Stores, Binary Only Module)
		\item Javascript ist: imperativ, objekt-prientiert, funktional, scripting Sprache (nicht funktional im Sinne von Haskell und Co, nicht oo wie C++ oder Java)
		\item Programmierer kann eigenen Programmierstiel wählen d.h. wenig Zwang der Sprache
		\item Problem: Javascript bietet wenig Garantien (keine Typsicherheit, wenig Abgrenzung der Module, Wenig Debug-Unterstützung, Wenig Optimierung)
		\item Lösung: Nachrüsten möglich (Typsystem: Typ-Annotation in Kommentaren dadurch besste Analyse), Optimierung(Compilation auf eine VM, Sprachoptimierung nach der Analyse (Compiler von Javascript nach Javascript), Debugging (Firebug, Browser Konsolen)))	
		\item Javascript als:
		\begin{itemize}
			\item imperative Sprache: Variablen, Einfaches Typsystem, Ablaufsemantik usw... aber keine sinnvolle Abgrenzung von imperativen Sprachen möglich da Variablen sind dynamisch getypt und anfänglich nicht kompiliert
			\item funktionale Sprache: Funktionen, Rekursion, Funktionen als first-class-objects, anonyme Funktionen aber keine Seiteneffekt-Freiheit, kein komplexes Typsystem und keine referentielle Transparenz
			\item Objekt-orientierte Sprache: Klassen, Vererbung und Polymorphismus aber keine Namensräume, keine Zugriffsmodifikatioren, kein typ-casting und keine Templates
		\end{itemize}
	\end{itemize}
	\subsection{Funktionale Aspekte}
	\begin{itemize}
		\item Zugriff auf Parameter über Namen oder Array-Arguments (variadische Funktionen)
		\item variadische Funktionen: Funktionen mit unbestimmter Arität (Zugriff oder Arraystruktur)
		\item Funktionsdeklaration und Rekursion wie üblich möglich
		\item Funktionen sind Werte (first-class-objects) d.h. können als Variablen gespeichert bzw. als Paramter übergeben werden
		\item Anonyme Funktionen Bsp: var quad = function (x) \{return x*x;\} (Funktion ist in quad-Variable gespeichert aber hat keinen Funktionsnamen)
		\item Closures: Anonyme Lambda-Definition mit freien Variablen (d.h. in der Funktion können Variablen genutzt werden welche im inneren Kontext nicht bekannt sind)
		\item Eingebauter Compiler: Funktions-Körper oder bel. Evaluation durch eval-Funktion
		\item Problem des Compilers: Langsam (da dynamisch) und Gefährlich (geringe Kontrolle über Seiten-Veränderung)
	\end{itemize}
	\subsection{Objekt-orientierte Aspekte}
	\begin{itemize}
		\item klassisch: Klassen-basierte Vererbung (Java,C++)
		\item in JS: Prototypen-basierte Vererbung
		\begin{itemize}
			\item Objekt besitzt Methode selber oder delegiert Suche an Prototyp
			\item Prototyp ist Eigenschaft der Klasse
			\item Methoden-Suche ist dynamisch und passiert zur Laufzeit
			\item Methoden können zur Laufzeit hinzugefügt werden
			\item Vererbungs-Hierarchie ensteht zur Laufzeit
			\item Vererbungs-Hierarchie kann zur Laufzeit dynamisch und mehrfach verändert werden
		\end{itemize}
		\item jede Funktionsauswertung läuft in einem Kontext (this)
		\item Globales Objekt = Window 
		\item Erzeugtes Objekt = Konstruktor
		\item Gebundenes Objekt mit call und apply
	\end{itemize}
	\subsection{Scripting-Aspekte}
	\begin{itemize}
		\item Dynamische Erweiterung: Attribute werden dynamisch generiert
		\item Bietet Möglichkeit, alle Instanzvariablen zu iterieren (for (fieldnName in object) \dots)
	\end{itemize}
	\subsection{Weitere Aspekte}
	\begin{itemize}
		\item Javascript Typen: schwach getypte Sprache (typeof prüft grundlegende Typen (primitiv oder Objekt), instanceof prüft auf Klasse eines Objetes) d.h. Variablen nicht getypt
		\item 3 Bestandteile von Javascript
		\begin{enumerate}
			\item ECMAScript = standarditisierte Sprachkern von Javascript (wesentliche Eigenschaften werden dort implemenziert)
			\item Document Object Model = Sprachunabhängige API für HTML (\& XML, CSS) Dokumente und Baumstrukturen, kompletter Zugriff auf alle HTML Tags (innerHTML), den Stiel (style und classname), Events via Event-Handler
			\item Browser Object Model = Stark Plattform-spezifische API für den Browser (window, document)
		\end{enumerate}
		\item Server-Side Javascript: Interpretation und Verarbeitung von Javascript auf Seiten am Server
		\item Weiter Entwicklungen:
		\begin{itemize}
			\item CoffeeScript = knappes Javascript, Postfix if Notation, Einrückungen (Python), Pattern Matching (Hasekll) $\rightarrow$ Compilation nach Javascript(Problem: erzeugter JS Code schwer lesbar und debugging aufwendig)
			\item TypeScript = Javascript für die große Systementwicklung, statisch getypte Variablen, Typ-Prüfung zur Compilezeit, echte Klassen usw
			\item Dart = Javascript in C und Java erweitert (bisschen wie TypeScript)ś
		\end{itemize}
	\end{itemize}
	\section{JSON}
	\begin{itemize}
		\item JSON = Javascript Object Notation
		\item Format für die Serialisierung von Objekten
		\item Unterstütze Datenstrukturen: Objekte, Arrays, Strings, Boolean, Integer, null
		\item Bewertung: kompakter als XML (weniger Overhead als XML, weniger aufwendiges Parsen), keine Attribute, keine DTD
		\item Einsatzbereiche: Server generiert JSON und Client generiert in JS aus JSON ein Objekt
		\item Risiko: Server sendet mehr als JSON Subset
		\item Erweiterung: JSONP (JSON + Padding)
		\item Idee: 
		\begin{itemize}
			\item Client fordert ein bestimmtes Padding vom Server an
			\item Server führt Padding durch und liefert JSON + Padding aus
			\item Ergebnis wird als dynamisches Script Tag dem Dokument hinzugefügt
		\end{itemize}
		\item Ergebnis: Keine Restriktion hinsichtlich XHR same-origin, da Script Tag; JSON wird an richtiger Stelle in Javascript eingebaut
	\end{itemize}
	\section{Javascript Pattern}
	\begin{itemize}
		\item ???
	\end{itemize}
	\section{Smallworld und Powerlaw Netzwerke}
	\subsection{Grundgedanke}
	\begin{itemize}
		\item Ausgangsexperiment: Paket soll an eine Person geschickt werden, ausgehend von einer ausgewählten Person die nur den Name und der Region kannte des Empfängerns
		\item Vorgehen: Senden an eine Person von der man glaubt dass diese die Zielperson kennt
		\item Ergebnis: Manche Pakete erreichten ihr Ziel nicht aber wenn ein Paket sein Ziel erreichte,
		dann im Schnitt mit weniger als 6 Zwischenschritten
		\item Analogie: Routing im Internet
		\item Ziel: Modellierung dieser Verhältnisse
	\end{itemize}
	\subsection{Erdös Netze}
	\begin{itemize}
		\item Erdös sehr aktiver Mathematik
		\item Erdöszahl:
		\begin{itemize}
			\item 0 = Erdös selber
			\item 1 = Koautor von Erdös (Person welche mit Erdös veröffentlicht hat)
			\item 2 = Koautor eines Koautor von Erdös 
			\item usw...
			\item Nicht jeder Mathematiker hat eine Erdös Zahl
		\end{itemize}
		\item Durchschnittliche Erdös-Zahl ist kleiner als 5
	\end{itemize}
	\subsubsection{Persone-Netze)}
	\begin{itemize}
		\item Bsp: Xing, LinkedIn
		\item Es gibt viele Leute, welche noch keine Kontakte eingetragen haben
		\item Sobald es einen bestätigten Kontakt gibt, existiert mit großer Wahrscheinlichkeit ein Pfad zu dieser Person über andere Personen
		\item Bezeichung: Small World Phänomen
	\end{itemize}
	\subsection{Systematische Charakterisierung von Small World}
	\begin{itemize}
		\item Problem = Unscharfe Definition (Wo beginnt Small World denn genau?)
		\item Grundidee: Die Welt sieht \glqq klein\grqq aus
		\item Idee:
		\begin{itemize}
			\item Welt soll potentiell groß sein 
			\item Sie ist aber zu meiner Überraschung doch recht klein weil ich bei der Bewegung immer mal wieder auf Bekannte von Bekannten stoße
		\end{itemize}
		\item Erstes Kriterium: Sinnvolle Gesamtsituation
		\begin{itemize}
			\item Knotenzahl viel größer als Kantenanzahl pro Knoten
			\item Kantenanzahl pro Knoten viel größer log(Knotenanzahl)
		\end{itemize}
		\item Clustering Koeffizient eines Knoten
		\begin{itemize}
			\item Frage: Welcher Anteil der möglichen Verbindungen der Nachbarn untereinander ist tatsächlch im Graphen realisiert?
			\item Nahe bei 1: Die Nachbarn eines kennen sich gut untereinander (B)
			\item Nahe bei 0: Die Nachbarn eines Knotens sind einander meist fremd (A)
			\item \glqq Wenn A den B kennt und A den C kennt, dann ist die Wahrscheinlichkeit hoch dass auch B und C miteinander zu tun haben
			\begin{center}
				\includegraphics[scale=3]{img/cluster_koeffizient.jpg}
			\end{center}
		\end{itemize}
		\item Typische Entfernung (Mittlere minimale Pfadlänge)
		\begin{itemize}
			\item Beschreibt die minimale Länge eines Weges zwischen zwei Knoten, gemittelt über alle verbundenen Knotenpaaren
		\end{itemize}
		\item Small World Charakterisierung
		\begin{itemize}
			\item Hoher Clusterkoeffizient: Gute,lokale Vernetzung
			\item geringe Pfadlänge: Hinreichend viele globale Brücken
			\item 6 degrees of separation-Eigenschaft: mit rein lokalem Wissen alleine eine gute,kurze Route finde
		\end{itemize}
	\end{itemize}
	\subsection{3 wichtige Netzklassen}
	\subsubsection{Allgemein}
	\begin{itemize}
		\item Idee zur Generierung: Initialer Graph durch Wachstumprozess werden neue Knoten und Kanten erzeugt, basierend auf einem Zufallsprozess
		\item 3 wichtige Graphklassen
		\begin{itemize}
			\item Erdös und Renyi (Gleichverteilung)
			\item Watts und Strogatz (Decaying Long Distance)
			\item Barabasi und Albert (Preferential Attachment)
		\end{itemize}
	\end{itemize}
	\subsubsection{Erdös und Renyi}
	\begin{itemize}
		\item Idee: Zufälliger Graph (Gleichverteilit)
		\item 2 Modelle:
		\begin{itemize}
			\item Zufälliger Graph: Unter allen möglichen Graphen mit n Knoten und e Kanten wähle einen zufällig auf Basis einer Gleichverteilung aus
			\item Zufällige Kante: Graph mit n Knoten, für jede Kante zwischen zwei Knoten wird mit der Wahrscheinlichkeit p realisiert
		\end{itemize}
		\item Verteilung der Knotengrade: Binomial-Verteilung
		\item Cluster Koeffizient: Recht klein
		\item Typische Entfernung: Abhängig von den Parametern
	\end{itemize}
	\subsubsection{Watts und Strogatz}
	\begin{itemize}
		\item Idee: Lokal regulärer Graph mit Fernverbindungen, Wahrscheinlichkeit der Fernverbindungen sinkt mit deren Länge 
		\item Lokale Konnektivität: jeder Knoten hat Kontakt zu allen Nachbarn der Schachbrettentfernung p
		\item Long Range Konnektivität: Jeder Knoten hat zu weiteren q Knoten Kontakt, Wahrscheinlichkeit ist dabei invers proportional zur Entfernug (Wahrscheinlichkeit = $\frac{1}{d^{r}}$)
	\end{itemize}
	\subsubsection{Albert und Barabasi}
	\begin{itemize}
		\item Idee: Preferential Attachment (Netz mit n Knoten, jeder neue Knoten verbindet sich zufällig mit alten Knoten, Wahrscheinlichkeit ist proportional zur Zahl der Kanten, die ein alter Knoten bereits hat)
		\item ALSO: Wer schon viele Verbindungen hat, kriegt noch besonders viele weitere (Monopole)
	\end{itemize}
	\subsection{Anwendungen im Web 2.0}
	\begin{itemize}
		\item Netzwerke mit Power Law Verhalten werden von 2 – 3 Big Players monopolisiert
		\item Folge: die Masse wird nicht gesehen und geht in der Bedeutungslosigkeit unter (Beispiel: Youtube)

	\end{itemize}
\end{document}